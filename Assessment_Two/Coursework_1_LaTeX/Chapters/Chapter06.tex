%************************************************
\chapter{Testing}\label{ch:testing} % $\mathbb{ZNR}$
%************************************************

The testing of the application was performed in two stages:

\begin{itemize}
\item In the early development stages unit-tests were written for the base classes.
\item After chaining the application parts together, the applications correct behaviour was mainly tested by using the application. This was necessary as unit-testing \ac{GUI} and multi-threaded code is extremely difficult.
\end{itemize}

\section{Unit-tests}
\label{sec:unit_tests}

The source-code of the written unit-tests can be seen in \autoref{ap:tests}.

To run the test-cases the \href{http://www.nunit.org/}{NUnit}-framework (\url{http://www.nunit.org/}) is needed.

The test-cases cover the passing of different types of parameters.

A short list of tests shall be provided to give a short overview:

\LTXtable{\textwidth}{Chapters/Chapter06_TestTable}

\section{Acceptance-tests}
\label{sec:acceptance_tests}

After testing the base classes the main testing was performed by running the application and entering different kinds of input:

Tests included the correct display of error-messages if a field containing a \ac{URL} was filled in incorrectly or if values were not provided that were needed (e.g. omitting a name for a favourite).

A list of performed tests:

\LTXtable{\textwidth}{Chapters/Chapter06_AccTestTable}

Even though this can not guarantee that the application is error free, it provides a good measure that it \textit{should} work reliably in most cases.
