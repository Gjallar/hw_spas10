%************************************************
\chapter{Requirement's checklist}\label{ch:requirements} % $\mathbb{ZNR}$
%************************************************

From the requirements stated in \autoref{sec:remit}, the following were fulfilled:

\begin{description}
\item[MSI01:] Implemented in StockItem class with getters and setters.
\item[MSI02:] Implemented in a manager-class that handles adding and deleting in-memory.
\item[MSI03:] Implemented in StockItem class.
\item[MSI04:] Implemented in AppDataManager-class.
\item[MBA01:] Implemented in BankAccount class with getters and setters.
\item[MBA02:] Fake-method for ordering: adjusts account balance, without transfer money.
\item[MBA03:] Implemented in AppDataManager-class.
\item[MDA01:] Implemented in FileHandler-class and StockItem-class.
\item[MDA02:] Implemented in FileHandler-class.
\item[MDA03:] Implemented in StockItem-class.
\item[MDA04:] Implemented in StockItem-class.
\item[MDA04:] Implemented in StockItem-class.
\item[MDA05:] Verified via testing.
\item[GUI01:] Implemented via WinForms.
\item[GUI02:] Implemented via WinForms.
\end{description}

Apart from fulfilling these requirements the following features were implemented as well to improve the user-experience of the program:

\begin{description}
\item[Error notification:] Upon entering invalid information the user will be informed about the mistakes by the \ac{GUI}.
\item[Bank account persistence:] It is possible to import and export bank accounts as well.
\item[Order quantity:] It is possible to order a certain quantity instead of always ordering the required number of items.
\end{description}