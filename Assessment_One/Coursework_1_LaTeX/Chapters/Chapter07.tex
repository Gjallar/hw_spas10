%************************************************
\chapter{Conclusions}\label{ch:conclusions} % $\mathbb{ZNR}$
%************************************************

After developing the application and looking at the resulting design, there are a few things that might have been accomplished differently and produced a better and cleaner design.

The data-binding part - due to the removal of the back-write mechanism (so the view does not automatically update the model) was mostly unnecessary and restricted the choices of collections in the model-part to \texttt{BindingLists}, whereas the model should actually be independent of the views.

Another problem arising from the \texttt{BindingLists} is that they can not be returned as read-only and this way can provide write access to the model. However, the lists were still implemented as public properties to allow unit-testing.

Another fact that became clear while testing was that the auto-creation of stock items and bank accounts in the model was unfortunate to test the correct behaviour of adding and deleting items. One could not easily add a specific item and delete the same one, but let the model create one, acquire a reference to it, delete it and test if the list was changed accordingly. A thorough \ac{TDD}-approach might have circumvented these shortcomings.

However, even with these shortcomings the application should fulfil the requirements.
Moreover, the implementation of the generic file handler should be reusable to persist all kinds of objects - even in other projects.

And with the provided test-cases many cases should be covered that should allow a thorough refactoring of the application if the need arises.